\documentclass[11pt]{article}
\usepackage[a4paper, margin=2.54cm]{geometry}
\usepackage[utf8]{inputenc}
\usepackage[spanish, mexico]{babel}
\usepackage[spanish]{layout}
\usepackage[article]{ragged2e}
\usepackage{textcomp}
\usepackage{amsmath}
\usepackage{amsfonts}

\title{Trabajo Práctico: Sistemas Operativos}
\author{Mellino, Natalia \and Farizano, Juan Ignacio}
\date{}

\begin{document}

\maketitle

\section*{Objetivo}
El objetivo es hacer un servidor de partidas de ta-te-ti. Este le permite a los 
usuarios conectarse, crear nuevas partidas y aceptarlas, así como unirse a una
partida iniciada o por iniciar como observador. Las distintas opciones entre las que
puede elegir el usuario, son las siguientes:

\begin{enumerate}
    \item Iniciar la comunicación con el servidor: para esto el usuario provee un
          nombre y si este se encuentra en uso el servidor lo acepta.
    \item Listar los juegos disponibles: se muestran tanto los juegos que están en
          desarrollo como los que están esperando un contrincante.
    \item Crear un nuevo juego: este será entre el jugador que lo cree y el primero
          que lo acepte.
    \item Realizar una jugada: el usuario puede realizar un movimiento en el 
          tablero o abandonar la partida si lo desea. El servidor devolverá una
          error si la jugada no es válida.
    \item Observar un juego: el usuario que pida observar un juego, comenzará a 
          recibir todos los cambios en el estado del juego.
    \item Dejar de observar un juego.
    \item Terminar la conexión: el usuario abandona todos los juegos en los que 
          participe y luego la conexión se termina.
\end{enumerate}

Al ejecutar el programa del cliente, este tendrá un menú interactivo con instrucciones
detalladas acerca de cómo realizar cada comando.

\section*{Ejecución del Cliente y Servidor}

Para ejecutar el Cliente, en la consola de Erlang debemos hacer primero \texttt{c(client).} y
luego, \texttt{client:start(IP, Port).} donde IP y Port deben ser la direccion de IP y
Puerto correspondiente. \\

Para ejecutar el Server, debemos cargar dos módulos: \texttt{c(tateti).} y \texttt{c(server).}, una
vez hecho esto, simplemente escribimos: \texttt{server:start().} y ya estará listo
para aceptar conexiones e interactuar con los clientes. \\

\subsection*{Observaciones:}

\begin{itemize}
    \item Para ejecutar el servidor se asume que los nodos en las PCs ya están
          conectados entre sí.
    \item Por la manera en que se intercambian los mensajes entre Cliente y Servidor 
          en nuestra implementación, es importante aclarar que para el funcionamiento
          correcto del Servidor, el Cliente debe estar programado en Erlang. 
\end{itemize}

\section*{Implementación}

\subsection*{Servidor}

Cuando se incializa el servidor se abre el socket y comienzan cuatro procesos:

\begin{itemize}
    \item \texttt{dispatcher}: es el proceso encargado de aceptar las conexiones de los clientes
          entrantes. Al aceptar la conexión de un cliente, se inicia un proceso llamado
          \texttt{pupdater} que se encargará de mandarle al cliente las actualizaciones acerca del
          estado de los juegos en los que participe. Luego, el \texttt{dispatcher} inicia
          un hilo \texttt{psocket} que es el que se encargará únicamente de atender los pedidos de dicho
          cliente.

          \hspace{0.5cm} El hilo \texttt{psocket}, primero se encarga de que el cliente proporcione un 
          nombre de usuario. Una vez que este ya tiene un nombre, puede comenzar a realizar
          pedidos. Cuando el cliente hace un pedido, se inicia el proceso \texttt{pcomando}
          que recibe el pedido que mandó el cliente y si no se produce ningún error, lo cumple
          y responde a la petición del cliente. 
    \item \texttt{master}: este proceso es el que guarda la información de los jugadores y 
          los juegos que están en curso. Cuando el proceso \texttt{pcomando} recibe la
          petición del cliente, este se la manda a \texttt{master}, que se encarga de 
          realizarla, actualizar su información (si es necesario) y luego responderle a \texttt{pcomando} con la información que éste
          le debe enviar de vuelta al cliente.
    \item \texttt{pbalance} y \texttt{pstat}: como podemos tener distintos servidores corriendo
          en distintos nodos, aparentando ser un único servidor, es necesario tener funciones
          que se encarguen de repartir el trabajo que debe realizar cada nodo, para asi evitar 
          que en un solo nodo se realicen muchas peticiones mientras que los otros no están trabajando.
          Para ello, el proceso \texttt{pstat} manda regularmente la información de carga del nodo en el que
          se encuentra al resto de los nodos. Esta información la recibe \texttt{pbalance} y se encarga de
          guardarla. Entonces, cuando un cliente hace una request, se le pide a \texttt{pbalance} que
          nos devuelva el nodo que menos carga tiene en ese momento, para así realizar la petición del 
          cliente en dicho nodo. 
\end{itemize}

En resumen, son estos procesos, junto con algunas funciones auxilares, las que conforman
y hacen funcionar nuestro servidor, estas tratan de ser lo más exhaustivas posibles al
momento de detectar si hubo algún error en la conexión o al procesar una petición para poder
evitar que éste se caiga.

\subsection*{Cliente}

Cuando inicializamos el Cliente con la función \texttt{start}, ésta comprueba primero
si no hubo ningún error al intentar conectarse. Si la conexión fue exitosa, se llama a la
función \texttt{sender}, que se encarga de que el Cliente proporcione un nombre de usuario. Una vez
que ya lo tiene, se inicia el proceso \texttt{receiver} y \texttt{sender} se vuelve a llamar
recursivamente. Ya en esta parte, el \texttt{sender} se encarga de enviar al Server todas 
las peticiones del Cliente. Por otro lado, el \texttt{receiver} es el que recibe las respuestas
por parte del Server y se encarga de mostrárselas al Cliente por pantalla.

Básicamente, la implementación del Cliente consta de un menú interactivo, donde a éste se le presentan
una serie de opciones para elegir, y cuando elige alguna de ellas, estas son convertidas a 
peticiones que luego serán mandadas por el \texttt{sender} al Server a través del Socket y el
\texttt{receiver} se encargará de comunicarle la respuesta del Servidor al Cliente. 

\subsection*{Tateti}

El módulo tateti, es tan sólo una implementación simple y práctica del juego tateti, podemos ver
que la función principal es \texttt{make\_play}, la cual se encarga de realizar la jugada
pedida por el usuario y modificar el tablero, siempre controlando que los movimientos 
que se realicen sean válidos.

\end{document}